\subsection{Part II: Determination and Interpretation of R Squared }

First the rows of all data are defined
\lstinputlisting[language=R, firstline=21, lastline=27]{../Q2_ii_iii.R}

Then data for each fleet categories are extracted using the following function
\lstinputlisting[language=R, firstline=11, lastline=19  ]{../Q2_ii_iii.R}

The regression model was created using the following function
\lstinputlisting[language=R, firstline=95, lastline=119]{../Q2_ii_iii.R}

The r squared value was obtained from the following function
\lstinputlisting[language=R, firstline=46, lastline=49]{../Q2_ii_iii.R}

\subsubsection*{R-Squared Summary for Salaries and Wages}
\begin{table}[htbp]
    \centering
    \begin{tabular}{lc}
        \toprule
        Category           & R-squared \\
        \midrule
        Small Narrowbodies & 0.784049  \\
        Large Narrowbodies & 0.7882288 \\
        Widebodies         & 0.8758896 \\
        Total Fleet        & 0.8334002 \\
        \bottomrule
    \end{tabular}
    \caption{R-squared values for different aircraft categories}
    \label{tab:rsquared}
\end{table}

The R-squared (\( R^2 \)) value is a statistical measure that represents the proportion of the variance for a dependent variable that's explained by an independent variable or variables in a regression model. An \( R^2 \) value closer to 1 indicates that the model explains a large proportion of the variance, while a value closer to 0 indicates less explanatory power.

\begin{itemize}
    \item \textbf{Small Narrowbodies:} The \( R^2 \) value is 0.784049, which means that approximately 78.4\% of the variability in the data for small narrowbody aircraft is explained by the model. This suggests a relatively strong fit for this category.

    \item \textbf{Large Narrowbodies:} The \( R^2 \) value for large narrowbody aircraft is 0.7882288, or about 78.8\%, indicating a similar level of explanatory power as that for small narrowbodies. This value suggests the model captures most of the variance in this category but with slightly better fit compared to small narrowbodies.

    \item \textbf{Widebodies:} With an \( R^2 \) value of 0.8758896, or approximately 87.6\%, the model explains a significant portion of the variance in the widebody aircraft data. This high value indicates a strong model fit for this category.

    \item \textbf{Total Fleet:} The \( R^2 \) value for the entire fleet is 0.8334002, or 83.3\%, suggesting that the model performs well in explaining the variance when considering all aircraft types collectively. This provides an overall assessment of model accuracy across all categories.

\end{itemize}