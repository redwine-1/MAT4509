\documentclass[a4paper,12pt]{article}
\usepackage{graphicx}
\usepackage{amsmath}
\usepackage{hyperref}
\usepackage{float}
\usepackage{booktabs}
\usepackage{caption}
\usepackage{geometry}
\usepackage{listings} 
\usepackage{xcolor}   
\usepackage{pgfplots} 
\usepackage{subfigure}
\pgfplotsset{compat=1.18} 
\geometry{margin=1in}

\title{Assignment 1: United Airlines Aircraft Operating Statistics Analysis}
\author{Muhtasim Redwan \\ Student ID: 22024002 \\ muhtasim.redwan.ae@proton.me}
\date{October 2024}

\begin{document}

\maketitle
\lstset{ %
    language=R,                     % Code language
    basicstyle=\ttfamily\footnotesize, % Code font
    numbers=left,                   % Line numbers on the left
    stepnumber=1,                   % Line number step size
    numbersep=5pt,                  % Space between line numbers and code
    backgroundcolor=\color{lightgray!20}, % Background color for code block
    showspaces=false,               % Show spaces
    showstringspaces=false,         % Don't show spaces in strings
    tabsize=4,                      % Tab size
    breaklines=true,                % Line breaking
    breakatwhitespace=false,        % Don't break at whitespaces only
    keywordstyle=\color{teal},        % Keywords color (calm but stands out)
    commentstyle=\color{olive},       % Comment color (subtle but readable)
    stringstyle=\color{magenta},
    numbers=none,
    frame=single
}

\section*{Introduction}
This report analyzes data from the ``United Airlines Aircraft Operating Statistics- Cost Per Block Hour (Unadjusted).xlsx" file
to answer several statistical questions. The analysis involves using statistical techniques
such as frequency distributions, measures of central tendency, measures of dispersion, and data visualization
through histograms, pie charts, bar diagrams, and box plots.
The data is processed using R programming language, and all visualizations and numerical outputs are based on the provided sample.

\section{Question 1}
The provided file contains data on small narrowbodies, large narrowbodies, widebodies and total feelt. Each fleet
type was analyzed individually. All data bounded by rectangle b2 to w158 was loaded using the following code:
\lstinputlisting[language=R, firstline = 6, lastline = 8]{../Q1_i_iii.R}

After that salary wages of small narrowbodies, large narrowbodies, widebodies and total fleet was extracted using the following code. where \textit{get\_data\_by\_row} was used to load data from the desired row.
\lstinputlisting[language=R, firstline = 10, lastline = 19]{../Q1_i_iii.R}

\subsection{Part I: The Regression Equation for a Single Dependent Variable}

In simple linear regression, the model is defined by the following equation:

\begin{equation}
    \hat{y} = \beta_0 + \beta_1 \text{PT} + \beta_2 \text{BPT} + \beta_3 \text{PDP} + \beta_4 \text{M} + \beta_5 \text{AO} + \beta_6 \text{I} + \beta_7 \text{DUA} + \epsilon
\end{equation}

where:
\begin{itemize}
    \item \(\hat{y}\): The predicted or estimated value of the dependent variable, representing the Salaries and Wages outcome of interest.
    \item \(\beta_0\): The intercept term, which is the expected value of \(\hat{y}\) when all independent variables are zero. This is a constant that shifts the regression model vertically.
    \item \(\beta_1, \beta_2, \dots, \beta_7\): The slope coefficients, where each \(\beta_i\) represents the change in the predicted value \(\hat{y}\) for a one-unit increase in the corresponding independent variable. These coefficients quantify the strength and direction of the relationship between each predictor and the dependent variable.
    \item \(\text{PT}\): Pilot Training, the cost or value associated with training pilots.
    \item \(\text{BPT}\): Benefits and Payroll Taxes, the expenses related to employee benefits and taxes.
    \item \(\text{PDP}\): Per Diem/Personnel, per diem costs for personnel.
    \item \(\text{M}\): Maintenance, the costs related to aircraft maintenance.
    \item \(\text{AO}\): Aircraft Ownership, the costs associated with owning the aircraft.
    \item \(\text{I}\): Indices, economic or operational indices affecting the wages.
    \item \(\text{DUA}\): Daily Utilization per Aircraft, the daily usage rate of each aircraft.
    \item \(\epsilon\): The error term or residual, which accounts for the variation in \(\hat{y}\) that cannot be explained by the linear relationship with the independent variables. It represents random deviations from the predicted values.
\end{itemize}

\subsection{Part II: Determination and Interpretation of R Squared }

First the rows of all data are defined
\lstinputlisting[language=R, firstline=21, lastline=27]{../Q2_ii.R}

Then data for each fleet categories are extracted using the following function
\lstinputlisting[language=R, firstline=10, lastline=18]{../Q2_ii.R}

The r squared value is calculated using the following function
\lstinputlisting[language=R, firstline=45, lastline=54]{../Q2_ii.R}

\subsubsection*{R-Squared Summary for Salaries and Wages}

\begin{table}[htbp]
    \centering
    \begin{minipage}{0.45\textwidth} % Adjust width as needed
        \centering
        \caption{R-Squared Summary for Salaries and Wages: Pilot Training Data}
        \begin{tabular}{@{}ll@{}}
            \toprule
            \textbf{Aircraft Type} & \textbf{R-Squared} \\ \midrule
            Small Narrowbodies     & 0.3626             \\
            Large Narrowbodies     & 0.5541             \\
            Widebodies             & 0.7685             \\
            Total Fleet            & 0.6107             \\ \bottomrule
        \end{tabular}
    \end{minipage}%
    \hspace{0.05\textwidth} % Space between the tables
    \begin{minipage}{0.45\textwidth}
        \centering
        \caption{R-Squared Summary for Salaries and Wages: Benefits and Payroll Taxes}
        \begin{tabular}{@{}ll@{}}
            \toprule
            \textbf{Aircraft Type} & \textbf{R-Squared} \\ \midrule
            Small Narrowbodies     & 0.2449             \\
            Large Narrowbodies     & 0.0582             \\
            Widebodies             & 0.0918             \\
            Total Fleet            & 0.1341             \\ \bottomrule
        \end{tabular}
    \end{minipage}
\end{table}

\begin{table}[htbp]
    \centering
    \begin{minipage}{0.45\textwidth} % Adjust width as needed
        \centering
        \caption{R-Squared Summary for Salaries and Wages: Per Diem/Personnel}
        \begin{tabular}{@{}ll@{}}
            \toprule
            \textbf{Aircraft Type} & \textbf{R-Squared} \\ \midrule
            Small Narrowbodies     & 0.4479             \\
            Large Narrowbodies     & 0.3492             \\
            Widebodies             & 0.0623             \\
            Total Fleet            & 0.3083             \\ \bottomrule
        \end{tabular}
    \end{minipage}%
    \hspace{0.05\textwidth} % Space between the tables
    \begin{minipage}{0.45\textwidth}
        \centering
        \caption{R-Squared Summary for Salaries and Wages: Maintenance}
        \begin{tabular}{@{}ll@{}}
            \toprule
            \textbf{Aircraft Type} & \textbf{R-Squared} \\ \midrule
            Small Narrowbodies     & 0.2398             \\
            Large Narrowbodies     & 0.0010             \\
            Widebodies             & 0.0458             \\
            Total Fleet            & 0.0630             \\ \bottomrule
        \end{tabular}
    \end{minipage}
\end{table}

\begin{table}[htbp]
    \centering
    \begin{minipage}{0.45\textwidth} % Adjust width as needed
        \centering
        \caption{R-Squared Summary for Salaries and Wages: Aircraft Ownership}
        \begin{tabular}{@{}ll@{}}
            \toprule
            \textbf{Aircraft Type} & \textbf{R-Squared} \\ \midrule
            Small Narrowbodies     & 0.4229             \\
            Large Narrowbodies     & 0.2501             \\
            Widebodies             & 0.1528             \\
            Total Fleet            & 0.1732             \\ \bottomrule
        \end{tabular}
    \end{minipage}
\end{table}
\subsubsection{Concise Interpretation of R-Squared Values}

The R-squared values reflect the extent to which various independent variables explain the variance in ``Salaries and Wages'':

\begin{enumerate}
    \item \textbf{Pilot Training Data:}
          Strong explanatory power, particularly for widebodies (0.7685), indicating that training data accounts for approximately 77\% of the variance in salaries and wages. The total fleet R-squared is also significant at 0.6107.

    \item \textbf{Benefits and Payroll Taxes:}
          Very low explanatory power, with small narrowbodies at 0.2449 and large narrowbodies at only 0.0582, suggesting minimal influence on salaries. The total fleet R-squared (0.1341) reinforces this finding.

    \item \textbf{Per Diem/Personnel:}
          Moderate explanatory power for small narrowbodies (0.4479) and large narrowbodies (0.3492), but limited impact for widebodies (0.0623) and overall (0.3083).

    \item \textbf{Maintenance:}
          Generally low across all categories, particularly for widebodies (0.0458), indicating minimal effect on salaries, with a total fleet R-squared of 0.0629.

    \item \textbf{Aircraft Ownership:}
          Moderate influence, especially for small narrowbodies (0.4229). R-squared values for large narrowbodies (0.2501) and widebodies (0.1528) are lower, with a total fleet R-squared of 0.1732.
\end{enumerate}




\subsection{Part III: Determination of  the Standard error }

The Standard error was calculated using the following function
\lstinputlisting[language=R, firstline=56, lastline=64]{../Q2_ii_iii.R}

\begin{table}[htbp]
    \centering
    \begin{minipage}{0.45\textwidth} % Adjust width as needed
        \centering
        \caption{Standard Errors for Salaries and Wages: Pilot Training Data}
        \begin{tabular}{@{}ll@{}}
            \toprule
            \textbf{Aircraft Type} & \textbf{Standard Error} \\ \midrule
            Small Narrowbodies     & 2.945658                \\
            Large Narrowbodies     & 1.897191                \\
            Widebodies             & 1.624223                \\
            Total Fleet            & 2.050287                \\ \bottomrule
        \end{tabular}
    \end{minipage}%
    \hspace{0.05\textwidth} % Space between the tables
    \begin{minipage}{0.45\textwidth}
        \centering
        \caption{Standard Errors for Salaries and Wages: Benefits and Payroll Taxes}
        \begin{tabular}{@{}ll@{}}
            \toprule
            \textbf{Aircraft Type} & \textbf{Standard Error} \\ \midrule
            Small Narrowbodies     & 0.3933256               \\
            Large Narrowbodies     & 0.4291333               \\
            Widebodies             & 0.4178515               \\
            Total Fleet            & 0.4008269               \\ \bottomrule
        \end{tabular}
    \end{minipage}
\end{table}

\begin{table}[htbp]
    \centering
    \begin{minipage}{0.45\textwidth} % Adjust width as needed
        \centering
        \caption{Standard Errors for Salaries and Wages: Per Diem/ Personnel}
        \begin{tabular}{@{}ll@{}}
            \toprule
            \textbf{Aircraft Type} & \textbf{Standard Error} \\ \midrule
            Small Narrowbodies     & 1.544458                \\
            Large Narrowbodies     & 1.678229                \\
            Widebodies             & 1.751502                \\
            Total Fleet            & 1.890908                \\ \bottomrule
        \end{tabular}
    \end{minipage}%
    \hspace{0.05\textwidth} % Space between the tables
    \begin{minipage}{0.45\textwidth}
        \centering
        \caption{Standard Errors for Salaries and Wages: Maintenance}
        \begin{tabular}{@{}ll@{}}
            \toprule
            \textbf{Aircraft Type} & \textbf{Standard Error} \\ \midrule
            Small Narrowbodies     & 0.7082815               \\
            Large Narrowbodies     & 0.327993                \\
            Widebodies             & 0.6431427               \\
            Total Fleet            & 0.5639674               \\ \bottomrule
        \end{tabular}
    \end{minipage}
\end{table}

\begin{table}[htbp]
    \centering
    \begin{minipage}{0.45\textwidth} % Adjust width as needed
        \centering
        \caption{Standard Errors for Salaries and Wages: Aircraft Ownership}
        \begin{tabular}{@{}ll@{}}
            \toprule
            \textbf{Aircraft Type} & \textbf{Standard Error} \\ \midrule
            Small Narrowbodies     & 0.3972191               \\
            Large Narrowbodies     & 0.1741028               \\
            Widebodies             & 0.1969666               \\
            Total Fleet            & 0.2725161               \\ \bottomrule
        \end{tabular}
    \end{minipage}
\end{table}
\subsection{Part IV: Correlation Matrix}

The Correlation matrix was created using the following function
\lstinputlisting[language=R, firstline=58, lastline=75]{../Q2_ii_iii.R}
\begin{figure}[H]
    \centering
    \subfigure[Small Narrowbodies]{
        \includegraphics[width=0.45\textwidth]{images/small_narrowbodies_corr_matrix.png}
    }
    \subfigure[Large Narrowbodies]{
        \includegraphics[width=0.45\textwidth]{images/large_narrowbodies_corr_matrix.png}
    }
    \subfigure[Widebodies]{
        \includegraphics[width=0.45\textwidth]{images/widebodies_corr_matrix.png}
    }
    \subfigure[Total Fleet]{
        \includegraphics[width=0.45\textwidth]{images/total_fleet_corr_matrix.png}
    }
    \caption{Correlation Matrix of Different Fleet Categories}
\end{figure}

\begin{table}[ht]
    \centering
    \begin{tabular}{|c|c|c|}
        \hline
        \textbf{Fleet Category} & \textbf{Strong Correlation}                                                                                    & \textbf{Weak Correlation}                                                                                               \\
        \hline
        Small Narrowbodies      & \begin{tabular}[c]{@{}c@{}} {pilot\_training} \\ {per\_diem\_personnel} \\ {aircraft\_ownership} \end{tabular} & \begin{tabular}[c]{@{}c@{}} {benefits} \\ {maintenance} \end{tabular}                                                   \\
        \hline
        Large Narrowbodies      & \begin{tabular}[c]{@{}c@{}} {pilot\_training} \\ {per\_diem\_personnel} \\ {aircraft\_ownership} \end{tabular} & \begin{tabular}[c]{@{}c@{}} {benefits} \\ {maintenance} \end{tabular}                                                   \\
        \hline
        Widebodies              & {pilot\_training}                                                                                              & \begin{tabular}[c]{@{}c@{}} {benefits} \\ {maintenance} \\ {aircraft\_ownership}\\ {per\_diem\_personnel} \end{tabular} \\
        \hline
        Total Fleet             & \begin{tabular}[c]{@{}c@{}} {pilot\_training} \\ {per\_diem\_personnel} \end{tabular}                          & \begin{tabular}[c]{@{}c@{}} {benefits} \\ {maintenance} \\ {aircraft\_ownership}\\ \end{tabular}                        \\
        \hline
    \end{tabular}
    \caption{Summary of Strong and Weak Correlations with {salary\_wages} Across Fleet Categories}
\end{table}
\subsection{Part v: Box Plot for Purchased Goods, Aircraft Ownership, and Daily Utilization}
Box plots are developed for the variables ``Purchased Goods", ``Aircraft Ownership", and ``Daily Utilization per Aircraft" to visualize their spread and identify any outliers. At first categories of each of the variables are defined.
\lstinputlisting[language=R, firstline = 11, lastline = 14]{../Q1_v.R}

Data for each variable was loaded using the \textit{get\_category\_data} function.

\lstinputlisting[language=R, firstline = 31, lastline = 39]{../Q1_v.R}

Box plot for each variable was drawn using the following functions.
\lstinputlisting[language=R, firstline = 41, lastline = 71]{../Q1_v.R}

\begin{figure}[H]
    \centering
    % First row, first image
    \subfigure[Small Narrowbodies]{
        \includegraphics[width=0.45\textwidth]{images/Aircraft_Ownership_small_narrowbodies.png}
    }
    % First row, second image
    \subfigure[Large Narrowbodies]{
        \includegraphics[width=0.45\textwidth]{images/Aircraft_Ownership_large_narrowbodies.png}
    }

    % Second row, first image
    \subfigure[Widebodies]{
        \includegraphics[width=0.45\textwidth]{images/Aircraft_Ownership_widebodies.png}
    }
    % Second row, second image
    \subfigure[Total Fleet]{
        \includegraphics[width=0.45\textwidth]{images/Aircraft_Ownership_total_fleet.png}
    }

    \caption{Box Plot of the variable ``Aircraft Ownership"}
\end{figure}
\begin{figure}[H]
    \centering
    % First row, first image
    \subfigure[Small Narrowbodies]{
        \includegraphics[width=0.45\textwidth]{images/Daily_Utilization_small_narrowbodies.png}
    }
    % First row, second image
    \subfigure[Large Narrowbodies]{
        \includegraphics[width=0.45\textwidth]{images/Daily_Utilization_large_narrowbodies.png}
    }

    % Second row, first image
    \subfigure[Widebodies]{
        \includegraphics[width=0.45\textwidth]{images/Daily_Utilization_widebodies.png}
    }
    % Second row, second image
    \subfigure[Total Fleet]{
        \includegraphics[width=0.45\textwidth]{images/Daily_Utilization_total_fleet.png}
    }

    \caption{Box Plot of the variable ``Daily Utilization per Aircraft"}
\end{figure}
\begin{figure}[H]
    \centering
    % First row, first image
    \subfigure[Small Narrowbodies]{
        \includegraphics[width=0.45\textwidth]{images/Purchased_Goods_small_narrowbodies.png}
    }
    % First row, second image
    \subfigure[Large Narrowbodies]{
        \includegraphics[width=0.45\textwidth]{images/Purchased_Goods_large_narrowbodies.png}
    }

    % Second row, first image
    \subfigure[Widebodies]{
        \includegraphics[width=0.45\textwidth]{images/Purchased_Goods_widebodies.png}
    }
    % Second row, second image
    \subfigure[Total Fleet]{
        \includegraphics[width=0.45\textwidth]{images/Purchased_Goods_total_fleet.png}
    }

    \caption{Box Plot of the variable ``Purchased Goods"}
\end{figure}
\subsection{Part VI: The Hypothesis on Each of the Independent Variables}
\begin{longtable}{|l|l|l|}
    \hline
    \textbf{Model} & \textbf{Variable}    & \textbf{p-value} \\ \hline
    \endfirsthead

    \hline
    \textbf{Model} & \textbf{Variable}    & \textbf{p-value} \\ \hline
    \endhead

    \hline
    \endfoot

    \hline
    \endlastfoot

    \textbf{Small Narrowbodies}
                   & pilot\_training      & 0.317            \\
                   & benefits             & 0.020            \\
                   & per\_diem\_personnel & 0.020            \\
                   & maintenance          & 0.873            \\
                   & aircraft\_ownership  & 0.095            \\ \hline

    \textbf{Large Narrowbodies}
                   & pilot\_training      & 0.013            \\
                   & benefits             & 0.170            \\
                   & per\_diem\_personnel & 0.004            \\
                   & maintenance          & 0.491            \\
                   & aircraft\_ownership  & 0.127            \\ \hline

    \textbf{Widebodies}
                   & pilot\_training      & 9.58e-06         \\
                   & benefits             & 0.709            \\
                   & per\_diem\_personnel & 0.003            \\
                   & maintenance          & 0.777            \\
                   & aircraft\_ownership  & 0.292            \\ \hline

    \textbf{Total Fleet}
                   & pilot\_training      & 0.002            \\
                   & benefits             & 0.206            \\
                   & per\_diem\_personnel & 0.002            \\
                   & maintenance          & 0.984            \\
                   & aircraft\_ownership  & 0.382            \\ \hline
\end{longtable}

Based on the p-values obtained from the regression models for the different aircraft categories, the following independent variables have a p-value greater than 0.05, indicating that they may not be statistically significant. These variables may be considered for removal from the model:

\begin{itemize}
    \item \textbf{Small Narrowbodies:}
          \begin{itemize}
              \item \textit{Pilot Training} (p-value =  0.317)
              \item \textit{Maintenance} (p-value = 0.8728)
              \item \textit{Aircraft Ownership} (p-value = 0.0951)
          \end{itemize}

    \item \textbf{Large Narrowbodies:}
          \begin{itemize}
              \item \textit{Benefits} (p-value = 0.1696)
              \item \textit{Maintenance} (p-value = 0.4906)
              \item \textit{Aircraft Ownership} (p-value = 0.1268)
          \end{itemize}

    \item \textbf{Widebodies:}
          \begin{itemize}
              \item \textit{Benefits} (p-value = 0.7096)
              \item \textit{Maintenance} (p-value = 0.7756)
              \item \textit{Aircraft Ownership} (p-value = 0.2919)
          \end{itemize}

    \item \textbf{Total Fleet:}
          \begin{itemize}
              \item \textit{Benefits} (p-value = 0.2058)
              \item \textit{Maintenance} (p-value = 0.9837)
              \item \textit{Aircraft Ownership} (p-value = 0.3818)
          \end{itemize}
\end{itemize}
\section*{Conclusion}
This report has provided a detailed statistical analysis of the United Airlines Aircraft Operating Statistics dataset. To create this report I have computed measures of central tendency, measures of dispersion, created frequency distributions, and visualized the data using various plots, each contributing valuable insights into the dataset.

\end{document}
