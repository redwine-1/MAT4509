\documentclass[a4paper,12pt]{article}
\usepackage{graphicx}
\usepackage{amsmath}
\usepackage{hyperref}
\usepackage{float}
\usepackage{booktabs}
\usepackage{caption}
\usepackage{geometry}
\usepackage{listings} 
\usepackage{xcolor}   
\usepackage{pgfplots} 
\usepackage{subfigure}
\usepackage{titlesec} 
\pgfplotsset{compat=1.18} 
\geometry{margin=1in}

\title{Assignment 1: United Airlines Aircraft Operating Statistics Analysis}
\author{Muhtasim Redwan \\ muhtasim.redwan.ae@proton.me}
\date{October 2024}

\begin{document}

\begin{titlepage}
    \centering
    \vspace*{1.5cm}

    % University/Institution Name
    {\LARGE Bangabandhu Sheikh Mujibur Rahman Aviation and Aerospace University (BSMRAAU)\par}
    \vspace{1.5cm}

    \vspace*{\fill} % Start vertical space
    % Title of the Assignment
    {\huge\bfseries United Airlines Aircraft Operating Statistics Analysis\par}
    \vspace{0.5cm}

    % Author's Name
    {\large Muhtasim Redwan\par}
    {\large Student ID: 22024002\par}
    \vspace{0.5cm}

    % Course Name
    {\large Course Name: Probability and Statistics\par}
    {\large Course Code: MAT 4509}
    \vspace{0.5cm}

    % Instructor's Name
    {\large Instructor: Assoc Prof Dr. M Siddikur Rahman\par}
    \vspace{0.5cm}

    \vspace*{\fill} % End vertical space

    % Date
    {\large \today\par}

    \vfill

    \includegraphics[width=0.25\textwidth]{university_logo.png}

    % Bottom of the page
    {\large Department of Aeronautical Engineering (Avionics)\par}
    {\large Year: 2024\par}
\end{titlepage}
\maketitle
\lstset{ %
    language=R,                     % Code language
    basicstyle=\ttfamily\footnotesize, % Code font
    numbers=left,                   % Line numbers on the left
    stepnumber=1,                   % Line number step size
    numbersep=5pt,                  % Space between line numbers and code
    backgroundcolor=\color{lightgray!20}, % Background color for code block
    showspaces=false,               % Show spaces
    showstringspaces=false,         % Don't show spaces in strings
    tabsize=4,                      % Tab size
    breaklines=true,                % Line breaking
    breakatwhitespace=false,        % Don't break at whitespaces only
    keywordstyle=\color{teal},        % Keywords color (calm but stands out)
    commentstyle=\color{olive},       % Comment color (subtle but readable)
    stringstyle=\color{magenta},
    numbers=none,
    frame=single
}

\section*{Introduction}
This report analyzes data from the ``United Airlines Aircraft Operating Statistics- Cost Per Block Hour (Unadjusted).xlsx" file
to answer several statistical questions. The analysis involves using statistical techniques
such as frequency distributions, measures of central tendency, measures of dispersion, and data visualization
through histograms, pie charts, bar diagrams, and box plots.
The data is processed using R programming language, and all visualizations and numerical outputs are based on the provided sample.

\section{Question 1}
The provided file contains data on small narrowbodies, large narrowbodies, widebodies and total feelt. Each fleet
type was analyzed individually. All data bounded by rectangle b2 to w158 was loaded using the following code:
\lstinputlisting[language=R, firstline = 6, lastline = 8]{../Q1_i_iii.R}

After that salary wages of small narrowbodies, large narrowbodies, widebodies and total fleet was extracted using the following code. where \textit{get\_data\_by\_row} was used to load data from the desired row.
\lstinputlisting[language=R, firstline = 10, lastline = 19]{../Q1_i_iii.R}

\subsection{Part i: Frequency Distribution of Salaries and Wages}

The frequency distribution of each type of fleet was created using \textit{get\_frequency\_distribution} function.
\lstinputlisting[language=R, firstline=31, lastline=65, numbers=none,  frame=single]{../Q1_i_iii.R}

\begin{table}[H]
    \centering
    \caption{Frequency Distribution of Salaries and Wages}

    \begin{minipage}{0.45\textwidth}
        \centering
        \caption*{Small Narrowbodies}
        \begin{tabular}{|c|c|}
            \hline
            Class Interval & Frequency \\
            \hline
            (256,328]      & 5         \\
            (328,400]      & 5         \\
            (400,472]      & 5         \\
            (472,544]      & 2         \\
            (544,616]      & 3         \\
            \hline
        \end{tabular}
    \end{minipage}%
    \hspace{0.05\textwidth} % Space between the two tables
    \begin{minipage}{0.45\textwidth}
        \centering
        \caption*{Large Narrowbodies}
        \begin{tabular}{|c|c|}
            \hline
            Class Interval & Frequency \\
            \hline
            (319,380]      & 3         \\
            (380,441]      & 7         \\
            (441,502]      & 1         \\
            (502,563]      & 4         \\
            (563,624]      & 3         \\
            \hline
        \end{tabular}
    \end{minipage}
\end{table}

\begin{table}[H]
    \centering
    \begin{minipage}{0.45\textwidth}
        \centering
        \caption*{Widebodies}
        \begin{tabular}{|c|c|}
            \hline
            Class Interval & Frequency \\
            \hline
            (526,637]      & 3         \\
            (637,748]      & 6         \\
            (748,859]      & 3         \\
            (859,970]      & 2         \\
            (970,1.08e+03] & 3         \\
            \hline
        \end{tabular}
    \end{minipage}%
    \hspace{0.05\textwidth} % Space between the two tables
    \begin{minipage}{0.45\textwidth}
        \centering
        \caption*{Total Fleet}
        \begin{tabular}{|c|c|}
            \hline
            Class Interval & Frequency \\
            \hline
            (347,424]      & 3         \\
            (424,501]      & 6         \\
            (501,578]      & 3         \\
            (578,655]      & 5         \\
            (655,732]      & 1         \\
            \hline
        \end{tabular}
    \end{minipage}
\end{table}

\subsection{Part II: Determination and Interpretation of R Squared }

First the rows of all data are defined
\lstinputlisting[language=R, firstline=21, lastline=27]{../Q2_ii.R}

Then data for each fleet categories are extracted using the following function
\lstinputlisting[language=R, firstline=10, lastline=18]{../Q2_ii.R}

The r squared value is calculated using the following function
\lstinputlisting[language=R, firstline=45, lastline=54]{../Q2_ii.R}

\subsubsection*{R-Squared Summary for Salaries and Wages}

\begin{table}[htbp]
    \centering
    \begin{minipage}{0.45\textwidth} % Adjust width as needed
        \centering
        \caption{R-Squared Summary for Salaries and Wages: Pilot Training Data}
        \begin{tabular}{@{}ll@{}}
            \toprule
            \textbf{Aircraft Type} & \textbf{R-Squared} \\ \midrule
            Small Narrowbodies     & 0.3626             \\
            Large Narrowbodies     & 0.5541             \\
            Widebodies             & 0.7685             \\
            Total Fleet            & 0.6107             \\ \bottomrule
        \end{tabular}
    \end{minipage}%
    \hspace{0.05\textwidth} % Space between the tables
    \begin{minipage}{0.45\textwidth}
        \centering
        \caption{R-Squared Summary for Salaries and Wages: Benefits and Payroll Taxes}
        \begin{tabular}{@{}ll@{}}
            \toprule
            \textbf{Aircraft Type} & \textbf{R-Squared} \\ \midrule
            Small Narrowbodies     & 0.2449             \\
            Large Narrowbodies     & 0.0582             \\
            Widebodies             & 0.0918             \\
            Total Fleet            & 0.1341             \\ \bottomrule
        \end{tabular}
    \end{minipage}
\end{table}

\begin{table}[htbp]
    \centering
    \begin{minipage}{0.45\textwidth} % Adjust width as needed
        \centering
        \caption{R-Squared Summary for Salaries and Wages: Per Diem/Personnel}
        \begin{tabular}{@{}ll@{}}
            \toprule
            \textbf{Aircraft Type} & \textbf{R-Squared} \\ \midrule
            Small Narrowbodies     & 0.4479             \\
            Large Narrowbodies     & 0.3492             \\
            Widebodies             & 0.0623             \\
            Total Fleet            & 0.3083             \\ \bottomrule
        \end{tabular}
    \end{minipage}%
    \hspace{0.05\textwidth} % Space between the tables
    \begin{minipage}{0.45\textwidth}
        \centering
        \caption{R-Squared Summary for Salaries and Wages: Maintenance}
        \begin{tabular}{@{}ll@{}}
            \toprule
            \textbf{Aircraft Type} & \textbf{R-Squared} \\ \midrule
            Small Narrowbodies     & 0.2398             \\
            Large Narrowbodies     & 0.0010             \\
            Widebodies             & 0.0458             \\
            Total Fleet            & 0.0630             \\ \bottomrule
        \end{tabular}
    \end{minipage}
\end{table}

\begin{table}[htbp]
    \centering
    \begin{minipage}{0.45\textwidth} % Adjust width as needed
        \centering
        \caption{R-Squared Summary for Salaries and Wages: Aircraft Ownership}
        \begin{tabular}{@{}ll@{}}
            \toprule
            \textbf{Aircraft Type} & \textbf{R-Squared} \\ \midrule
            Small Narrowbodies     & 0.4229             \\
            Large Narrowbodies     & 0.2501             \\
            Widebodies             & 0.1528             \\
            Total Fleet            & 0.1732             \\ \bottomrule
        \end{tabular}
    \end{minipage}
\end{table}
\subsubsection{Concise Interpretation of R-Squared Values}

The R-squared values reflect the extent to which various independent variables explain the variance in ``Salaries and Wages'':

\begin{enumerate}
    \item \textbf{Pilot Training Data:}
          Strong explanatory power, particularly for widebodies (0.7685), indicating that training data accounts for approximately 77\% of the variance in salaries and wages. The total fleet R-squared is also significant at 0.6107.

    \item \textbf{Benefits and Payroll Taxes:}
          Very low explanatory power, with small narrowbodies at 0.2449 and large narrowbodies at only 0.0582, suggesting minimal influence on salaries. The total fleet R-squared (0.1341) reinforces this finding.

    \item \textbf{Per Diem/Personnel:}
          Moderate explanatory power for small narrowbodies (0.4479) and large narrowbodies (0.3492), but limited impact for widebodies (0.0623) and overall (0.3083).

    \item \textbf{Maintenance:}
          Generally low across all categories, particularly for widebodies (0.0458), indicating minimal effect on salaries, with a total fleet R-squared of 0.0629.

    \item \textbf{Aircraft Ownership:}
          Moderate influence, especially for small narrowbodies (0.4229). R-squared values for large narrowbodies (0.2501) and widebodies (0.1528) are lower, with a total fleet R-squared of 0.1732.
\end{enumerate}




\subsection{Part iii: Histogram of Grouped Salaries}
A histogram is created  based on the frequency distribution calculated in part i
using \textit{plot\_histogram} function.

\lstinputlisting[language=R, firstline=99, lastline=110]{../Q1_i_iii.R}

\begin{figure}[H]
    \centering
    \subfigure[Small Narrowbodies]{
        \includegraphics[width=0.45\textwidth]{images/histogramSmallNarrowBodies.png}
    }
    \subfigure[Large Narrowbodies]{
        \includegraphics[width=0.45\textwidth]{images/histogramLargeNarrowBodies.png}
    }
    \subfigure[Widebodies]{
        \includegraphics[width=0.45\textwidth]{images/histogramWidebodies.png}
    }
    \subfigure[Total Fleet]{
        \includegraphics[width=0.45\textwidth]{images/histogramTotalFleet.png}
    }
    \caption{Histogram of Sallary Wages of Diffent Fleet Categories}
\end{figure}


\subsection{Part iv: Pie Chart and Bar Diagram for Maintenance and Load Factor}
The variable ``Maintenance" had three categories. At first \textit{maintenance\_categories} and \textit{maintenance\_rows} are defined.
\lstinputlisting[language=R, firstline = 11, lastline = 12]{../Q1_iv.R}

``Maintenance" data of each feelt categories was loaded using \textit{get\_maintenace\_category} function.
\lstinputlisting[language=R, firstline = 21, lastline = 30]{../Q1_iv.R}

The pie chart was drawn using the following code:
\lstinputlisting[language=R, firstline = 53, lastline = 64]{../Q1_iv.R}

\begin{figure}[H]
    \centering
    % First row, first image
    \subfigure[Small Narrowbodies]{
        \includegraphics[width=0.45\textwidth]{images/small_narrowbodies_maintenace_pie.png}
    }
    % First row, second image
    \subfigure[Large Narrowbodies]{
        \includegraphics[width=0.45\textwidth]{images/large_narrowbodies_maintenace_pie.png}
    }

    % Second row, first image
    \subfigure[Widebodies]{
        \includegraphics[width=0.45\textwidth]{images/widebodies_maintenace_pie.png}
    }
    % Second row, second image
    \subfigure[Total Fleet]{
        \includegraphics[width=0.45\textwidth]{images/total_fleet_maintenace_pie.png}
    }

    \caption{Pie Chart of the variable ``Maintenance"}
\end{figure}

The bar charts were drawn using the following the code:
\lstinputlisting[language=R, firstline = 67, lastline = 78]{../Q1_iv.R}

\begin{figure}[H]
    \centering
    \subfigure[Small Narrowbodies]{
        \includegraphics[width=0.45\textwidth]{images/small_narrowbodies_load_factor_bar.png}
    }
    \subfigure[Large Narrowbodies]{
        \includegraphics[width=0.45\textwidth]{images/large_narrowbodies_load_factor_bar.png}
    }
    \subfigure[Widebodies]{
        \includegraphics[width=0.45\textwidth]{images/widebodies_load_factor_bar.png}
    }
    \subfigure[Total Fleet]{
        \includegraphics[width=0.45\textwidth]{images/total_fleet_load_factor_bar.png}
    }
    \caption{Bar Chart of the variable ``Load Factor"}
\end{figure}

\subsection{Part v: Box Plot for Purchased Goods, Aircraft Ownership, and Daily Utilization}
Box plots are developed for the variables ``Purchased Goods", ``Aircraft Ownership", and ``Daily Utilization per Aircraft" to visualize their spread and identify any outliers. At first categories of each of the variables are defined.
\lstinputlisting[language=R, firstline = 11, lastline = 14]{../Q1_v.R}

Data for each variable was loaded using the \textit{get\_category\_data} function.

\lstinputlisting[language=R, firstline = 31, lastline = 39]{../Q1_v.R}

Box plot for each variable was drawn using the following functions.
\lstinputlisting[language=R, firstline = 41, lastline = 71]{../Q1_v.R}

\begin{figure}[H]
    \centering
    % First row, first image
    \subfigure[Small Narrowbodies]{
        \includegraphics[width=0.45\textwidth]{images/Aircraft_Ownership_small_narrowbodies.png}
    }
    % First row, second image
    \subfigure[Large Narrowbodies]{
        \includegraphics[width=0.45\textwidth]{images/Aircraft_Ownership_large_narrowbodies.png}
    }

    % Second row, first image
    \subfigure[Widebodies]{
        \includegraphics[width=0.45\textwidth]{images/Aircraft_Ownership_widebodies.png}
    }
    % Second row, second image
    \subfigure[Total Fleet]{
        \includegraphics[width=0.45\textwidth]{images/Aircraft_Ownership_total_fleet.png}
    }

    \caption{Box Plot of the variable ``Aircraft Ownership"}
\end{figure}
\begin{figure}[H]
    \centering
    % First row, first image
    \subfigure[Small Narrowbodies]{
        \includegraphics[width=0.45\textwidth]{images/Daily_Utilization_small_narrowbodies.png}
    }
    % First row, second image
    \subfigure[Large Narrowbodies]{
        \includegraphics[width=0.45\textwidth]{images/Daily_Utilization_large_narrowbodies.png}
    }

    % Second row, first image
    \subfigure[Widebodies]{
        \includegraphics[width=0.45\textwidth]{images/Daily_Utilization_widebodies.png}
    }
    % Second row, second image
    \subfigure[Total Fleet]{
        \includegraphics[width=0.45\textwidth]{images/Daily_Utilization_total_fleet.png}
    }

    \caption{Box Plot of the variable ``Daily Utilization per Aircraft"}
\end{figure}
\begin{figure}[H]
    \centering
    % First row, first image
    \subfigure[Small Narrowbodies]{
        \includegraphics[width=0.45\textwidth]{images/Purchased_Goods_small_narrowbodies.png}
    }
    % First row, second image
    \subfigure[Large Narrowbodies]{
        \includegraphics[width=0.45\textwidth]{images/Purchased_Goods_large_narrowbodies.png}
    }

    % Second row, first image
    \subfigure[Widebodies]{
        \includegraphics[width=0.45\textwidth]{images/Purchased_Goods_widebodies.png}
    }
    % Second row, second image
    \subfigure[Total Fleet]{
        \includegraphics[width=0.45\textwidth]{images/Purchased_Goods_total_fleet.png}
    }

    \caption{Box Plot of the variable ``Purchased Goods"}
\end{figure}
\subsection{Part vi: Interpretation of the Plots}
\subsubsection{Histogram of Sallary Wages of Diffent Fleet Categories}

\begin{enumerate}
    \item \textbf{Small Narrowbodies:} The distribution is more uniform for the first three intervals and then declines.
    \item \textbf{Large Narrowbodies:} This has a left-skewed distribution, with a high frequency in the second interval followed by a sharp decline.
    \item \textbf{Widebodies:} The distribution is more spread out and somewhat bimodal, with two intervals showing higher frequencies and smaller peaks in between.
    \item \textbf{Total Fleet:} The distribution is skewed to the left, with a peak in the second interval.
\end{enumerate}

\subsubsection{Pie Chart of the variable ``Maintenance"}
`Burden' is consistently the largest or second largest slice in all categories, with Widebodies showing the largest Burden slice among all fleets.
`Third Party' costs are substantial across all categories, particularly in Widebodies and Small Narrowbodies.
`Materials' and `Labor' contribute smaller portions relative to Burden and Third Party in all categories.
The pie plot for Widebodies shows the most extreme proportions, with Burden and Third Party clearly dominating, while Small Narrowbodies shows a more balanced distribution between `Burden' and `Third Party costs'.

\subsubsection{Bar Chart of the variable “Load Factor”}
All fleet categories show positive growth from 1995 to 2015.
Small Narrowbodies and Large Narrowbodies show a more consistent upward trend, while Widebodies exhibit a more fluctuating pattern, with a dip and recovery around 2008.
The Total Fleet represents a collective performance that mirrors the trends of the individual fleet types, with a clear growth pattern from 1995 to 2015, peaking at 0.8375 in 2011.

\subsubsection{Box Plots}

Box plots, also known as box-and-whisker plots, provide a visual summary of the distribution of a dataset. They are particularly useful for identifying the spread, skewness, and the presence of outliers in the data. The main components of a box plot are the minimum, first quartile (Q1), median (Q2), third quartile (Q3), and maximum, with additional features like whiskers and outliers

\section*{Conclusion}
This report has provided a detailed statistical analysis of the United Airlines Aircraft Operating Statistics dataset. To create this report I have computed measures of central tendency, measures of dispersion, created frequency distributions, and visualized the data using various plots, each contributing valuable insights into the dataset.

\end{document}
