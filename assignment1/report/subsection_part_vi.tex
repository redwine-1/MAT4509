\subsection{Part vi: Interpretation of the Plots}
\subsubsection{Histogram of Sallary Wages of Diffent Fleet Categories}

\begin{enumerate}
    \item \textbf{Small Narrowbodies:} The distribution is more uniform for the first three intervals and then declines.
    \item \textbf{Large Narrowbodies:} This has a left-skewed distribution, with a high frequency in the second interval followed by a sharp decline.
    \item \textbf{Widebodies:} The distribution is more spread out and somewhat bimodal, with two intervals showing higher frequencies and smaller peaks in between.
    \item \textbf{Total Fleet:} The distribution is skewed to the left, with a peak in the second interval.
\end{enumerate}

\subsubsection{Pie Chart of the variable ``Maintenance"}
`Burden' is consistently the largest or second largest slice in all categories, with Widebodies showing the largest Burden slice among all fleets.
`Third Party' costs are substantial across all categories, particularly in Widebodies and Small Narrowbodies.
`Materials' and `Labor' contribute smaller portions relative to Burden and Third Party in all categories.
The pie plot for Widebodies shows the most extreme proportions, with Burden and Third Party clearly dominating, while Small Narrowbodies shows a more balanced distribution between `Burden' and `Third Party costs'.

\subsubsection{Bar Chart of the variable “Load Factor”}
All fleet categories show positive growth from 1995 to 2015.
Small Narrowbodies and Large Narrowbodies show a more consistent upward trend, while Widebodies exhibit a more fluctuating pattern, with a dip and recovery around 2008.
The Total Fleet represents a collective performance that mirrors the trends of the individual fleet types, with a clear growth pattern from 1995 to 2015, peaking at 0.8375 in 2011.

\subsubsection{Box Plots}

Box plots, also known as box-and-whisker plots, provide a visual summary of the distribution of a dataset. They are particularly useful for identifying the spread, skewness, and the presence of outliers in the data. The main components of a box plot are the minimum, first quartile (Q1), median (Q2), third quartile (Q3), and maximum, with additional features like whiskers and outliers
